\chapter{Trigonometry}
\section{Functions recap}
Last week:
\begin{enumerate}
  \item Discussed concept of a function (modern concept).
  3 pieces of information:
  \begin{enumerate}
    \item
    $f: A \to B, f(x) = \frac{1}{1+x^2}$ \\
    Where A is the domain, B is the range, and what follows is the definition of
    the function.
  \end{enumerate}
\end{enumerate}

\section{Introduction to Trigonometry}
Now we look at a particular class of functions, \emph{trigonometric functions}.
Many physics and engineering applications require the use of trigonometric
functions.

%TODO: include fig 1
include fig 1

Observation: $\frac{a}{b} = \frac{c}{d}$. \\
This argument holds iff $ad = bc$ \\
It follows:
\begin{align}
  \frac{a}{\sqrt{a^2+b^2}} :& \\
  \frac{a^2}{a^2+b^2} \\
  &= \frac{a^2/b^2}{a^2/b^2 + 1} \\
  &= \frac{c^2/d^2}{c^2/d^2 + 1} \\
  &= \frac{c^2}{c^2+d^2}
  \intertext{also}
  \frac{a}{\sqrt{a^2+b^2}} &= \frac{c}{\sqrt{c^2+d^2}}
  \intertext{similarly}
  \frac{b}{\sqrt{a^2+b^2}} &= \frac{d}{\sqrt{c^2 + d^2}}
  \intertext{Ratios for corresponding sides depend only on the interior angle
  $\alpha$. Rectangular triangle completely characterized by one of its
  non-trivial interior angles.}
  \text{TODO: include fig 2}
\end{align}

Some stuff

\begin{align}
  \intertext{Define:(unambiguously)}
  \cos \alpha &= \frac{b}{\sqrt{a^2+b^2}} & \quad \frac{\text{adjacent}}{\text{hyptoenuse}} \\
  \sin \alpha &= \frac{a}{\sqrt{a^2+b^2}} & \quad \frac{\text{opposite}}{\text{hyptoenuse}} \\
  \tan \alpha &= \frac{a}{b} & \quad \frac{\text{opposite}}{\text{adjacent}} \\
\end{align}

\section{The Unit Circle}
TODO include fig 3

\begin{align}
  \sec{\beta} &= \frac{1}{\cos \beta} = \frac{1}{\xi} \\
  \csc{\beta} &= \frac{1}{\sin \beta} = \frac{1}{\eta} \\
  \cot{\beta} &= \frac{1}{\tan \beta}
\end{align}

include fig 4

\begin{align}
  \sin : \mathbb{R} \to \mathbb{R}, \text{Range}(\sin) = [-1, +1] \\
  \cos : \mathbb{R} \to \mathbb{R}, \text{Range}(\cos) = [-1, +1] \\
  \tan(\alpha) : \mathbb{R} \to \mathbb{R}: \frac{\sin \alpha}{\cos \alpha}
  \intertext{$\cos \alpha$ cannot be zero, so we exclude $\pi/2$ and $-\pi/2 and
  any multiple thereof$.}
  \tan(\alpha) : \mathbb{R} \setminus
    \left \{ \frac{\pi}{2} + k \pi : k \in \mathbb{Z} \right \} \to \mathbb{R}\\
  \cot(\theta) : \frac{\cos \theta}{\sin \theta} = \frac{\xi}{\eta}
  \intertext{any multiple of $\pi$ should not be in the domain because
  $\sin\theta$ will be a zero denominator}
  \cot \theta : \mathbb{R} \setminus
    \left \{ \frac{\pi}{2} + k \pi : k \in \mathbb{Z} \right \} \to \mathbb{R}\\
\end{align}

\section{Pythagoras}
$\xi^2 + \eta^2 = 1$ leads to \\
$\cos^2(\gamma) + \sin^2(\gamma) = 1$ \\
This equation is Pythagoras' theorem applied to the unit circle.

Divide by $\sin^2(\gamma)$:
\begin{align}
  \frac{\cos^2\gamma}{\sin^2 \gamma} +1 &= \frac{1}{\sin^2 \gamma} \\
  \cot^2 \gamma + 1 &= \csc^2 \gamma
  \intertext{Divide by $\cos^2\gamma (\neq 0)$}
  1 + \tan^2 \gamma = \sec^2 \gamma
\end{align}

\section{Radians}
include fig 5

tabularize tab 1

fig 6

tabularize tab 2

\section{Amplitude and Period}
\subsection{Period}
A period may be one full revolution ($2\pi$) around the unit circle, to bring
you back to your starting point.
\begin{align}
  \sin(x+2\pi) &= \sin(x) \\
  \sin(x+2 n \pi) &= \sin(x) : n \in mathbb{Z}\\
  \cos(x + 2\pi) &= \cos(x) \\
  % fig 7
  \intertext{Rotate by half revolution:}
  \cos(\theta + \pi) &= -\cos(\theta) \\
  \sin(\theta + \pi) &= -\sin(\theta)
  \intertext{This implies the tangent has period $\pi$.}
  \because \tan(\theta + \pi)
    &= \frac{\sin(\theta+\pi)}{\cos(\theta+\pi)} \\
    &= \frac{-\sin(\theta)}{-\cos(\theta)} \\
    &= \frac{\sin(\theta)}{\cos(\theta)} \\
    &= \tan(\theta) \\
  % fig 8
  \intertext{Rotate by quarter revolution:}
  \sin(\theta + \frac{\pi}{2}) &= x = \cos(\theta) \\
  \cos(\theta + \frac{\pi}{2}) &= -y = -\sin(\theta)
  \intertext{Reflections}
  %fig 9
  \cos(-\theta) &= \cos(\theta) \\
  \sin(-\theta) &= -y = -\sin(\theta)
\end{align}

\subsection{Some examples:}
\begin{align}
  \sin(\theta - \frac{\pi}{2}) &= \sin(- (\frac{\pi}{2} -\theta) ) \
  &= -\sin(\frac{\pi}{2}-\theta) \\
  &= -\sin(-\theta + \frac{\pi}{2}) \\
  &= -\cos(-\theta)\\
  &= -\cos(\theta)
\end{align}

%fig 10
%\begin{align}
%  \sin(2x) &= \sin(x + x) \\
%  &= \sin(x)\cos(x) + \cos(x)\sin(x) \\
%  &= 2\sin(x)\cos(x) \\
%  \cos(2x) &= \cos(x+x) \\
%    &= \cos(x)\cos(x) -\sin(x)\sin(x) \\
%    &= \cos^2(x) - \sin^2(x) \\
%    &= 1-2\sin^2(x) \label{trig1} \\
%    &= 2\cos^2(x)-1 \label{trig2} \\
%  (1) \cos(\alpha) &= 1-2\sin^2(\frac{\alpha}{2}) \\
%  &= \sin^2(\frac{\alpha}{2}) = \cos(\frac) \\ %TODO LOOK ME UP
%  (2) %TODO: look me up\\
%  \intertext{Using double angle identities:}
%  \sin(x) &= \frac{2\tan(x/2)}{1+tan^2(x/2)} \\
%  \cos(x) &= \frac{1-\tan^2(x/2)}{1+\tan^2(x/2)}
%\end{align}

\subsection{Examples:}
%\begin{align}
%  f(x) &= 5\cos(x)-12\sin(x)
%  \intertext{Rewrite}
%  f(x) &= A\cos(x\beta) \\
%  \intertext(Solve)
%  f(x) &= 15 = A\cos(x\beta) \\
%  \intertext{Need a point on the unit circle, so we scale to fit to r=1. To do
%  this, we divide and multiply by appropriate numbers}
%  f(x) &= \sqrt{5^2+(-12)^2} \left(\frac{5}{\sqrt{5^2+(-12)^2}} \cos x
%    - \frac{12}{\sqrt{5^2+(-12)^2)}}\sin x\right) \\
%  % TODO: fig 11
%  &= 13\left( \cos\beta\cos x - \sin\beta\sin x \right) \\
%  &= 13\cos(\beta +x )
%  \intertext{Solve}
%    f(x) &= 13\cos(\beta+x) = 15 \\
%    \cos(\beta +x) &= \frac{15}{13} > 1 \\
%    & \therefore \quad \text{no solutions}
%\end{align}

%\begin{align}
%  \intertext{Find exact value of}
%  \cos(\frac{5\pi}{24}) & \\
%  \frac{5\pi}{24} &= \frac{1}{2}\left(\frac{\pi}{4}+\frac{\pi}{6}\right) \\
%  \cos^2(\frac{5\pi}{24}) &= \cos^2\left(\frac{1}{2}\left(\frac{\pi}{4}+\frac{\pi}{6}\right)\right) \\
%  &= \frac{1}{2}\left( 1 + \cos(\pi/4)\cos(\pi/6) - \sin(\pi/4)\sin(\pi/6) \right) \\
%  \cos^2(\frac{5\pi}{24}) &= \frac{1}{2} ( 1 + \sqrt{2}/2 \sqrt{3}/2 - \sqrt{2}/2\sqrt{1}{2})
%\end{align}
%
